%%%%%%%%%%%%%%%%%%%%%%%%%%%%%%%%%%%%%%%%%
% Lachaise Assignment
% LaTeX Template
% Version 1.0 (26/6/2018)
%
% This template originates from:
% http://www.LaTeXTemplates.com
%
% Authors:
% Marion Lachaise & François Févotte
% Vel (vel@LaTeXTemplates.com)
%
% License:
% CC BY-NC-SA 3.0 (http://creativecommons.org/licenses/by-nc-sa/3.0/)
% 
%%%%%%%%%%%%%%%%%%%%%%%%%%%%%%%%%%%%%%%%%

%----------------------------------------------------------------------------------------
%	PACKAGES AND OTHER DOCUMENT CONFIGURATIONS
%----------------------------------------------------------------------------------------

\documentclass{article}

\input{structure.tex} % Include the file specifying the document structure and custom commands

%----------------------------------------------------------------------------------------
%	ASSIGNMENT INFORMATION
%----------------------------------------------------------------------------------------

\title{MAE001: Modelagem Matemática em Finanças I} % Title of the assignment
\date{Universidade Federal do Rio de Janeiro (UFRJ) --- \today} % University, school and/or department name(s) and a date

\author{Ramon Duarte de Melo\\ \texttt{ramonduarte@poli.ufrj.br}} % Author name and email address


%----------------------------------------------------------------------------------------

\begin{document}

\maketitle % Print the title

%----------------------------------------------------------------------------------------
%	INTRODUCTION
%----------------------------------------------------------------------------------------

\section*{Introdução} % Unnumbered section

O objetivo do Projeto 4 é implementar, avaliar e comparar o modelo do \emph{Capital Asset Pricing Model} com os dados fornecidos pelo mundo real, realizando comparações de estruturas de termo e produzindo gráficos com tais observações. 

Para tal, foi utilizada a linguagem \emph{Python 3.6.7}, com os módulos \emph{numpy} (métodos numéricos), \emph{pandas} (manipulação de dados), \emph{scipy} (fórmulas científicas) e \emph{matplotlib.pyplot} (visualização de dados).

Os dados utilizados para a confecção dos gráficos foram obtidos através do site do Tesouro Nacional (\textsl{tesouro.gov.br}). Os procedimentos de execução estão descritos num arquivo \texttt{.ipynb}, que requer o módulo \emph{Jupyter Notebook} para ser executado.


O código utilizado neste trabalho, bem como o deste relatório e as imagens geradas, foi aberto e disponibilizado publicamente no repositório https://github.com/ramonduarte/mmftrab4.


%----------------------------------------------------------------------------------------
%	PROBLEM 1
%----------------------------------------------------------------------------------------

\section*{Atividade \emph{1}} % Numbered section

Nesta atividade, a estrutura a termo da taxa de juros de títulos prefixados do Tesouro Nacional (\emph{LTNs}, sem prêmios semestrais) foi adquirida da web e dela extraída a taxa de juros dos anos de 2016, 2017 e 2018 para os próximos 6 a 7 anos.
Como alguns títulos não são emitidos no início do ano (sobretudo os de mais longa duração até o exercício), as datas foram padronizadas para o mesmo dia em cada ano, 2º de fevereiro.
A escolha por esta data deu-se por ser o primeiro dia útil onde todos os títulos foram disponibilizados.

Os arquivos foram obtidos em formato \texttt{.xls} (padrão Microsoft Excel 2003), parseados e plotados em gráficos $t \times r$.
Para cada gráfico, uma curva \emph{spline} cúbica que perpassa todos os pontos também foi plotada.


\begin{figure}[]
	\includegraphics[width=0.6\linewidth]{Figure_0.png}
	\centering
	
	\caption{Gráfico $t \times r$ para ativos \textbf{LTN} com data de emissão 02/02/2016.}
	\label{}
\end{figure}

\begin{figure}[]
	\includegraphics[width=0.6\linewidth]{Figure_1.png}
	\centering
	
	\caption{Gráfico $t \times r$ para ativos \textbf{LTN} com data de emissão 02/02/2017.}
	\label{}
\end{figure}

\begin{figure}[]
	\includegraphics[width=0.6\linewidth]{Figure_2.png}
	\centering
	
	\caption{Gráfico $t \times r$ para ativos \textbf{LTN} com data de emissão 02/02/2018.}
	\label{}
\end{figure}

%----------------------------------------------------------------------------------------
%	PROBLEM 2
%----------------------------------------------------------------------------------------


\section*{Atividade \emph{b}}

Para esta atividade, o processo utilizado foi o mesmo do da anterior.
As únicas mudanças foram:

\begin{itemize}
	\item Uma subbiblioteca do \emph{matplotlib}, externa ao \emph{pyplot}, teve de ser utilizada para gerar o gráfico em 3D, porque o \emph{pyplot} não gera gráficos 3D interativos, necessários para a escolha da melhor perspectiva.
	\item Todas as opções encontradas para o ativo \textbf{PETR4 ON} foram utilizadas, desde que estivessem precificadas após o fechamento do pregão da Bolsa de Valores de São Paulo em 06/06/2019.
\end{itemize}

\begin{figure}[H]
	\includegraphics[width=\linewidth]{Figure_3.png}
	\centering
	
	\caption{\textbf{Superfície de volatilidade}: séries $K \times \sigma$ separadas por data de exercício.}
	\label{}
\end{figure}

%----------------------------------------------------------------------------------------
%	PROBLEM 2
%----------------------------------------------------------------------------------------

\section*{Atividade c}

Com exceção da \emph{Figura 3}, foi possível enxergar o \emph{smile} em todos os gráficos plotados.
Como o ativo \textbf{PETR4 ON} é bastante negociado na bolsa de valores, suas distribuições tendem a se aproximarem do previsto pelo modelo teórico.
É justamente por isso que, na \emph{Figura 3}, a observação do \emph{smile} é mais difícil, pois as opções com exercício nesta data - mais distante - são menos negociadas e, portanto, não formam pontos suficientes para a construção da curva com visibilidade.

Na \emph{Figura 4}, inclusive, é possível notar a formação de várias curvas deste tipo ao longo das datas de exercício, bem como seu rareamento conforme o aumento do prazo.


\end{document}
